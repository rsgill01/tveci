\documentclass[12pt]{article}
\usepackage{graphics}
\usepackage{latexsym}
\usepackage{amssymb}
\usepackage{amsbsy}
\usepackage{verbatim}
\usepackage{url}

\begin{document}

\begin{center}
{\Large\bf Vignette for {\tt c.tveci} and {\tt r.tveci}}
\end{center}

This example demonstrates how to run code for fitting a threshold vector error correction (TVEC) model which includes data imputation for modeling minimum gasoline prices in a city based on crude oil prices.  

The data set used to illustrate the code includes these prices for the city of Perth each day in 2015.  The data set {\tt PerthMin2015.dat} has four columns {\tt date}, {\tt MinPrice}, {\tt CP}, and {\tt TodayCPAvailable}.  The column {\tt date} gives the date in {\em yyyymmdd} format, {\tt MinPrice} gives the lowest gas price in Perth on that date, {\tt CP} is the crude oil price on that date, and {\tt TodayCPAvailable} is an indicator variable which is 1 if the crude oil price was available on the date and 0 if not.  The following commands can be used to read the data set into R and to view the last few lines.
\begin{verbatim}
> perthData=as.matrix(read.table("PerthMin2015.dat", sep="\t",
+ header=TRUE))
> tail(perthData,15)
           date MinPrice    CP TodayCPAvailable
[351,] 20151217    124.9 34.95                1
[352,] 20151218    121.9 34.73                1
[353,] 20151219    118.9  0.00                0
[354,] 20151220    115.9  0.00                0
[355,] 20151221    112.9 34.74                1
[356,] 20151222    134.9 36.14                1
[357,] 20151223    127.9 37.50                1
[358,] 20151224    125.9 38.10                1
[359,] 20151225    121.9  0.00                0
[360,] 20151226    118.9  0.00                0
[361,] 20151227    114.9  0.00                0
[362,] 20151228    111.9 36.81                1
[363,] 20151229    129.9 37.87                1
[364,] 20151230    127.9 36.60                1
[365,] 20151231    124.9 37.04                1
\end{verbatim}

The source code for five R functions are freely available in the file {\tt Rcode.R}.  It can be loaded into R with the following command.
\begin{verbatim}
> source("Rcode.R")
\end{verbatim}
The function {\tt r.tvec} fits a TVEC model that allows for additional explantory variables if available in the matrix {\tt otherX}.  The function {\tt c.tvec} is a wrapper function for C code that is equivalent to {\tt r.tvec}, but faster since it is written in C rather than R.  The function {\tt predict.tvec} accepts the fitted TVEC model and input variables needed to predict the minimum gasoline prices on the next day.
The function {\tt r.tveci} implements the novel imputation procedure for estimating minimum gasoline prices based on the TVEC model by imputing missing crude oil prices.  The function {\tt c.tveci} is equivalent but uses the {\tt c.tvec} function internally instead of the {\tt r.tvec} function for fitting intermediate TVEC models so again it is faster because the computationally intense parts use C code instead of R code.

Some of the computationally intense parts are written in C to improve the speed of the code.  The C source code is freely available in the file {\tt tvec.c}.  
On a 64-bit Windows machine, the C library can be loaded using the following command.
\begin{verbatim}
> dyn.load("tvec.dll")
\end{verbatim}
For users who prefer not to use pre-complied code, information on using and creating shared objects and DLL files from source code is described in section 5.3 of the manual {\it Writing R Extensions} available at 

\noindent \url{https://cran.r-project.org/doc/manuals/r-release/R-exts.html}

\noindent and the links therein.

The following command shows how to use {\tt c.tveci} to fit the TVEC model with data imputation.  
\begin{verbatim}
> c.tveci.model=c.tveci(perthData[,2:4],otherX=NULL,lag=7)
\end{verbatim}
Estimates of the parameters $\beta$, $\gamma$, and $A$ can be extracted as follows.
\begin{verbatim}
> c.tveci.model$beta
[1] 2.485031
> c.tveci.model$gamma
[1] -23.49134
> c.tveci.model$A
              [,1]          [,2]
 [1,]  0.657464776 -0.0640037550
 [2,] 26.964071997 -1.3834669567
 [3,]  0.045929340  0.0883909500
 [4,] -1.598539043 -0.5158844018
 [5,]  0.250834801  0.0364460943
 [6,] -0.162630564 -1.0535092885
 [7,] -0.157445711  0.0354086570
 [8,] -2.381661579 -0.3275854288
 [9,] -0.174165647  0.0006437405
[10,]  0.701938881 -0.4317512551
[11,] -0.555751888 -0.0300257600
[12,]  0.280600450 -0.5417526773
[13,] -0.309200769  0.0075029184
[14,] -0.051480041 -0.3783098821
[15,]  0.150411011 -0.0445297302
[16,] -0.574459435 -0.5333856631
[17,]  0.007588942  0.0008712315
[18,] -0.300923104 -0.0069614959
[19,] -0.475463000 -0.0122684091
[20,]  0.172415463 -0.1499350877
[21,] -0.430621787 -0.0034398839
[22,] -0.156951525 -0.0618227198
[23,] -0.385253952 -0.0185141341
[24,] -0.109217591  0.1807023351
[25,] -0.359273852  0.0092258659
[26,] -0.097996782  0.0572361988
[27,] -0.319527890 -0.0065896080
[28,]  0.088564853 -0.0455894004
[29,] -0.282338380 -0.0176420312
[30,] -0.096759720 -0.0098054168
[31,]  0.483957744 -0.0063457632
[32,]  0.004916119 -0.0534337235
\end{verbatim}
As part of the output, the data set with the imputed missing values are also stored in a matrix {\tt imputed}.  The last fifteen lines for this matrix in the output with this data set are shown below.
\begin{verbatim}
> tail(c.tveci.model$imputed,15)
       MinPrice       CP
[351,]    124.9 34.95000
[352,]    121.9 34.73000
[353,]    118.9 34.87144
[354,]    115.9 34.67435
[355,]    112.9 34.74000
[356,]    134.9 36.14000
[357,]    127.9 37.50000
[358,]    125.9 38.10000
[359,]    121.9 37.92839
[360,]    118.9 38.62688
[361,]    114.9 38.54207
[362,]    111.9 36.81000
[363,]    129.9 37.87000
[364,]    127.9 36.60000
[365,]    124.9 37.04000
\end{verbatim}
The function {\tt r.tveci} gives identical results.  However, the computation time on a Dell 3.4 GHz workstation running Windows 10 is 23 seconds for {\tt r.tveci} but only 5 seconds for {\tt c.tveci}.

Finally, we can use the results for the fitted model stored in {\tt c.tveci.model} to predict the change in minimum gas price and the change in crude oil price on January 1, 2016 using the function {\tt predict.tveci}.
\begin{verbatim}
> predict.tveci(c.tveci.model)
            [,1]
[1,] -3.17819595
[2,] -0.07977937
\end{verbatim}
So, the minimum gas price is predicted to decrease by $3.18$ cents; the actual price decrease from December 31, 2015 to January 1, 2016 turned out to be $5.5$ cents.  (No crude oil price was reported on January 1, 2016 since it was a holiday.)
\end{document}
